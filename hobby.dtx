% \iffalse meta-comment
%<*internal>
\def\nameofplainTeX{plain}
\ifx\fmtname\nameofplainTeX\else
  \expandafter\begingroup
\fi
%</internal>
%<*install>
\input docstrip.tex
\keepsilent
\askforoverwritefalse
\preamble
Copyright (C) 2012 by Andrew Stacey
-------------------------------------------

This file may be distributed and/or modified under the
conditions of the LaTeX Project Public License, either version 1.3
of this license or (at
 your option) any later version.
The latest version of this license is in:

   http://www.latex-project.org/lppl.txt

and version 1.3 or later is part of all distributions of LaTeX
version 2005/12/01 or later.
\endpreamble
\generate{\file{tikzlibraryhobby.code.tex} {\from{hobby.dtx}{tikzlibrary}}}
\generate{\file{pgflibraryhobby.code.tex} {\from{hobby.dtx}{pgflibrary}}}
\generate{\file{hobby.code.tex}
{\from{hobby.dtx}{hobby}}}
%</install>
%<install>\endbatchfile
%<*internal>
\generate{
  \file{\jobname.ins}{\from{\jobname.dtx}{install}}
}
\ifx\fmtname\nameofplainTeX
  \expandafter\endbatchfile
\else
  \expandafter\endgroup
\fi
%</internal>
%<*driver>
\documentclass{l3doc}
\usepackage[T1]{fontenc}
\usepackage{csquotes}
\usepackage{lmodern}
\usepackage{tikz}
\usepackage{amsmath}
\usetikzlibrary{hobby,decorations.pathreplacing}
\usepackage[margin=3cm]{geometry}
\EnableCrossrefs
\CodelineIndex
\RecordChanges
\tikzset{
  shape example/.style={
    color=black!30,
    draw,
    fill=yellow!30,
    line width=.5cm,
    inner xsep=2.5cm,
    inner ysep=0.5cm}
}
\begin{document}
  \DocInput{\jobname.dtx}
\end{document}
%</driver>
% \fi
%
% \title{Hobby's Algorithm in TikZ/PGF}
% \author{Andrew Stacey}
% \date{2012-05-15}
% \maketitle
%
% 
% \section{Usage}
%
% The package is provided in form of a PGF library.
% It can be loaded with
%\begin{verbatim}
%\usetikzlibrary{hobby}
%\end{verbatim}
% or
%\begin{verbatim}
%\usepgflibrary{hobby}
%\end{verbatim}
% or the equivalent ConTeXt commands.
%
% The TikZ library installs a \Verb+to path+ which draws a smooth curve through the given points.
% \tikzset{
%   show curve controls/.style={
%    decoration={
%      show path construction,
%      curveto code={
%      \draw [blue, dashed]
%          (\tikzinputsegmentfirst)    -- (\tikzinputsegmentsupporta)
%          node [at end, draw, solid, red, inner sep=2pt]{};
%        \draw [blue, dashed]
%          (\tikzinputsegmentsupportb) -- (\tikzinputsegmentlast)
%          node [at start, draw, solid, red, inner sep=2pt]{};
%      }
%    },decorate
%  },
% }
%
% \begin{tikzpicture}
% \draw[scale=.1,postaction=show curve controls,line width=1mm,red] (0,0)
% .. controls (26.76463,-1.84543) and (51.4094,14.58441) .. (60,40)
% .. controls (67.09875,61.00188) and (59.76253,84.57518) .. (40,90)
% .. controls (25.35715,94.01947) and (10.48064,84.5022) .. (10,70)
% .. controls (9.62895,58.80421) and (18.80421,49.62895) .. (30,50);
% 
% \fill[green] (0,0) circle[radius=2pt]
%  (6,4) circle[radius=2pt]
%  (4,9) circle[radius=2pt]
%  (1,7) circle[radius=2pt]
%  (3,5) circle[radius=2pt];
% \draw[postaction=show curve controls,thick] (0,0) to[curve through={(6,4) .. (4,9) .. (1,7)}] (3,5);
% 
% \end{tikzpicture}
% 
% 
% \section{Implementing Hobby's Algorithm}
% 
% We start with a list of \(n+1\) points, \(z_0, \dotsc, z_n\).
% The base code assumes that these are already stored in two property lists: the \(x\)--coordinates in \Verb+\l_hobby_pointsx_prop+ and the \(y\)--coordinates in \Verb+\l_hobby_pointsy_prop+.
% The number, \(n\), is stored as \Verb+\l_hobby_npoints_prop+.
% As our arrays are \(0\)--indexed, the actual number of points is one more than this.
%
% From this we compute the distances and angles between successive points, storing these again as property lists.
% These are \Verb+\l_hobby_distances_prop+ and \Verb+\l_hobby_angles_prop+.
% The term indexed by \(k\) is the distance (or angle) of the line between the \(k\)th point and the \(k+1\)th point.
% For the internal nodes\footnote{Hobby calls the specified points \emph{knots}}, we store the difference in the angles in \Verb+\l_hobby_psi_prop+.
% The \(k\)th value on this is the angle subtended at the \(k\)th node.
% Thus (for an open path) this is indexed from \(1\) to \(n-1\).
% 
% The bulk of the work consists in setting up a linear system to compute the angles of the control points.
% At a node, say \(z_i\), we have various pieces of information:
%
% \begin{enumerate}
% \item The angle of the incoming curve, \(\phi_i\), relative to the straight line from \(z_{i-1}\) to \(z_i\)
% \item The angle of the outgoing curve, \(\theta_i\), relative to the straight line from \(z_i\) to \(z_{i+1}\)
% \item The tension of the incoming curve, \(\overline{\tau}_i\)
% \item The tension of the outgoing curve, \(\tau_i\)
% \item The speed of the incoming curve, \(\sigma_i\)
% \item The speed of the outgoing curve, \(\rho_i\)
% \end{enumerate}
%
% The tensions are known at the start.
% The speeds are computed from the angles.
% Thus the key thing to compute is the angles.
% This is done by imposing a ``mock curvature'' condition.%
% The formula for the mock curvature is:
%^^A
% \[
%   \hat{k}(\theta,\phi,\tau,\overline{\tau}) = \tau^2 \left( \frac{2(\theta + \phi)}{\overline{\tau}} - 6\theta\right)
% \]
%^^A
% and the condition that the mock curvatures have to satisfy is that at each \emph{internal} node, the curvatures must match:
% %
% \[
%   \hat{k}(\phi_i,\theta_{i-1},\overline{\tau}_i,\tau_{i-1})/d_{i-1} = \hat{k}(\theta_i,\phi_{i+1},\tau_i,\overline{\tau}_{i+1})/d_i.
% \]
%^^A
% Substituting in yields:
%^^A
% \[
%   \frac{\overline{\tau}_i^2}{d_{i-1}} \left( \frac{2(\phi_i + \theta_{i-1})}{\tau_{i-1}} - 6\phi_i\right) = \frac{\tau_i^2}{d_i} \left( \frac{2(\theta_i + \phi_{i+1})}{\overline{\tau}_{i+1}} - 6\theta_i \right).
% \]
%^^A
% Let us rearrange that to the following:
%^^A
% \begin{align*}
%   d_i \overline{\tau}_{i+1} \overline{\tau}_i^2 &\theta_{i-1} \\
%^^A
% +
% d_i \overline{\tau}_{i+1} \overline{\tau}_i^2 (1 - 3 \tau_{i-1}) &\phi_i \\
%^^A
% -
%  d_{i-1} \tau_{i-1} \tau_i^2 (1 - 3 \overline{\tau}_{i+1}) &\theta_i \\
%^^A
% -
%  d_{i-1} \tau_{i-1} \tau_i^2 &\phi_{i+1} \\
%^^A
% =
% 0
% \end{align*}
%^^A
% For an open path, this holds for \(i=1\) to \(i=n-1\).
% 
% We also have the condition that \(\theta_i + \phi_i = -\psi_i\) where \(\psi_i\) is the angle subtended at a node by the lines to the adjacent nodes.
% This holds for the internal nodes.
% Therefore for \(i=1\) to \(i=n-2\), the above simplifies to the following:
% %
% \begin{align*}
%   d_i \overline{\tau}_{i+1} \overline{\tau}_i^2 &\theta_{i-1} \\
% +
% (d_i \overline{\tau}_{i+1} \overline{\tau}_i^2 (3 \tau_{i-1} - 1)
% +
%  d_{i-1} \tau_{i-1} \tau_i^2 (3 \overline{\tau}_{i+1} - 1)) &\theta_i \\
% +
%  d_{i-1} \tau_{i-1} \tau_i^2 & \theta_{i+1} \\
% =
% - d_i \overline{\tau}_{i+1} \overline{\tau}_i^2 (3 \tau_{i-1} - 1) &\psi_i \\
% - d_{i-1} \tau_{i-1} \tau_i^2& \psi_{i+1}
% \end{align*}
% 
% We have two more equations.
% One involves \(\theta_0\).
% The other is the above for \(i = n-1\) with additional information regarding \(\psi_n\).
% It may be that one or either of \(\theta_0\) or \(\psi_n\) is specified in advance.
% In that case, the first equation is simply setting \(\theta_0\) to that value and the last equation involves substituting the value for \(\psi_n\) into the above.
% If not, they are given by formulae involving ``curl'' parameters \(\chi_0\) and \(\chi_n\) and result in the equations:
% %
% \begin{align*}
% \theta_0 &= \frac{\tau_0^3 + \chi_0 \overline{\tau}_1^3(3 \tau_0 - 1)}{\tau_0^3(3 \overline{\tau}_1 - 1) + \chi_0 \overline{\tau}_1^3} \phi_1 \\
% \phi_n &= \frac{\overline{\tau}_n^3 + \chi_n \tau_{n-1}^3(3 \overline{\tau}_n - 1)}{\overline{\tau}_n^3(3 \tau_{n-1} - 1) + \chi_n \tau_{n-1}^3} \theta_{n-1}
% \end{align*}
% %
% Using \(\phi_1 = - \psi_1 - \theta_1\), the first rearranges to:
% %
% \[
% (\tau_0^3(3 \overline{\tau}_1 - 1) + \chi_0 \overline{\tau}_1^3) \theta_0 + (\tau_0^3 + \chi_0 \overline{\tau}_1^3(3 \tau_0 - 1)) \theta_1 = - (\tau_0^3 + \chi_0 \overline{\tau}_1^3(3 \tau_0 - 1)) \psi_1.
% \]
% %
% The second should be substituted in to the general equation with \(i = n-1\).
% This yields:
% %  
% \begin{align*}
%   d_{n-1} \overline{\tau}_{n} \overline{\tau}_{n-1}^2 &\theta_{n-2} \\
% +
% (d_{n-1} \overline{\tau}_{n} \overline{\tau}_{n-1}^2 (3 \tau_{n-2} - 1)
% +
%  d_{n-2} \tau_{n-2} \tau_{n-1}^2 (3 \overline{\tau}_{n} - 1) \\
% - d_{n-2} \tau_{n-2} \tau_{n-1}^2  \frac{\overline{\tau}_n^3 + \chi_n \tau_{n-1}^3(3 \overline{\tau}_n - 1)}{\overline{\tau}_n^3(3 \tau_{n-1} - 1) + \chi_n \tau_{n-1}^3}) & \theta_{n-1} \\
% =
% - d_{n-1} \overline{\tau}_{n} \overline{\tau}_{n-1}^2 (3 \tau_{n-2} - 1) &\psi_{n-1}
% \end{align*}
% 
% The result is a linear system with an \(n \times n\) tridiagonal coefficient matrix.
% It is more natural to index the entries from \(0\).
% Let us write \(A_i\) for the subdiagonal, \(B_i\) for the main diagonal, and \(C_i\) for the superdiagonal.
% Let us write \(D_i\) for the target vector.
% Then we have the following formulae:
% %
% \begin{align*}
% A_i &= d_i \overline{\tau}_{i+1} \overline{\tau}^2_i \\
% B_0 &= \begin{cases}
% 1 & \text{if}\; \theta_0\; \text{given} \\
% \tau_0^3(3 \overline{\tau}_1 - 1) + \chi_0 \overline{\tau}^3_1 & \text{otherwise}
% \end{cases} \\
% B_i &= d_i \overline{\tau}_{i+1} \overline{\tau}_i^2 (3 \tau_{i-1} - 1) + d_{i-1} \tau_{i-1} \tau_i^2(3 \overline{\tau}_{i+1} - 1) \\
% B_{n-1} &= \begin{cases} d_{n-1} \overline{\tau}_{n} \overline{\tau}_{n-1}^2 (3 \tau_{n-2} - 1) + d_{n-2} \tau_{n-2} \tau_{n-1}^2(3 \overline{\tau}_{n} - 1) & \text{if}\; \phi_n\; \text{given} \\
%  d_{n-1} \overline{\tau}_{n} \overline{\tau}_{n-1}^2 (3 \tau_{n-2} - 1) + d_{n-2} \tau_{n-2} \tau_{n-1}^2(3 \overline{\tau}_{n} - 1)
% \\
% - d_{n-2} \tau_{n-2} \tau_{n-1}^2  \frac{\overline{\tau}_n^3 + \chi_n \tau_{n-1}^3(3 \overline{\tau}_n - 1)}{\overline{\tau}_n^3(3 \tau_{n-1} - 1) + \chi_n \tau_{n-1}^3}) & \text{otherwise}
% \end{cases} \\
% C_0 &= \begin{cases}
% 0 & \text{if}\; \theta_0\; \text{given} \\
% \tau_0^3 + \chi_0 \overline{\tau}_1^3(3\tau_0 - 1) & \text{otherwise}
% \end{cases} \\
% C_i &= d_{i-1} \tau_{i-1} \tau_i^2 \\
% D_0 &= \begin{cases}
% \overline{\theta}_0 & \text{if}\; \theta_0\; \text{given} \\
%  - (\tau_0^3 + \chi_0 \overline{\tau}_1^3(3 \tau_0 - 1)) \psi_1 & \text{otherwise}
% \end{cases} \\
% D_i &= - d_i \overline{\tau}_{i+1} \overline{\tau}_i^2 (3 \tau_{i-1} - 1) \psi_i
% - d_{i-1} \tau_{i-1} \tau_i^2 \psi_{i+1} \\
% D_{n-1} &= \begin{cases}
% - d_{n-1} \overline{\tau}_{n} \overline{\tau}_{n-1}^2 (3 \tau_{n-2} - 1) \psi_{n-1} - d_{n-2} \tau_{n-2} \tau_{n-1}^2 \overline{\phi}_n & \text{if}\; \phi_n\; \text{given} \\
% - d_{n-1} \overline{\tau}_{n} \overline{\tau}_{n-1}^2 (3 \tau_{n-2} - 1) \psi_{n-1} & \text{otherwise}
% \end{cases}
% \end{align*}
% 
% The next step in the implementation is to compute these coefficients and store them in appropriate property lists.
% Having done that, we need to solve the resulting tridiagonal system.
% This is done by looping through the arrays doing the following substitutions (starting at \(i = 1\)):
% %
% \begin{align*}
% B_i' &= B_{i-1}' B_i - A_i C_{i-1}' \\
% C_i' &= B_{i-1}' C_i \\
% D_i' &= B_{i-1}' D_i - A_i D_{i-1}'
% \end{align*}
% %
% followed by back-substitution:
% %
% \begin{align*}
% \theta_{n-1} &= D_{n-1}'/B_{n-1}' \\
% \theta_i &= (D_i' - C_i' \theta_{i+1})/B_i'
% \end{align*}
% 
% 
% This leaves us with the values for \(\theta_i\).
% We now substitute these into Hobby's formulae for the lengths:
% %
% \begin{align*}
% \rho_i &= \frac{2 + \alpha_i}{1 + (1 - c) \cos \theta_i + c \cos \phi_{i+1}} \\
% \sigma_{i+1} &= \frac{2 - \alpha_i}{1 + (1 - c) \cos \phi_{i+1} + c \cos \theta_i} \\
% \alpha_i &= a (\sin \theta_i - b \sin \phi_{i+1})(\sin \phi_{i+1} - b \sin \theta_i)(\cos \theta_i - \cos \phi_{i+1})
% \end{align*}
% %
% where \(a = \sqrt{2}\), \(b = 1/16\), and \(c = (3 - \sqrt{5})/2\).
% 
% Now \(\theta_i\) is the angle relative to the line from \(z_i\) to \(z_{i+1}\), so to get the true angle we need to add back that angle.
% Fortunately, we stored those angles at the start.
% So the control points are:
% %
% \begin{gather*}
%   (\rho_i \cos (\theta_i + \omega_i), \rho_i \sin (\theta_i + \omega_i)) + z_i \\
% (-\sigma_i \cos(\theta_i + \omega_i), -\sigma_i \sin(\theta_i + \omega_i)) + z_{i+1}
% \end{gather*}
% 
% 
% \begin{tikzpicture}
% \fill[green] (0,0) circle[radius=2pt]
%  (6,4) circle[radius=2pt]
%  (4,9) circle[radius=2pt]
%  (1,7) circle[radius=2pt]
%  (3,5) circle[radius=2pt];
% \draw[postaction=show curve controls,thick] (0,0) to[curve through={(6,4) .. (4,9) .. (1,7)}] (3,5);
% 
% \draw[scale=.1,postaction=show curve controls,thick,blue] (0,0)
% .. controls (26.76463,-1.84543) and (51.4094,14.58441) .. (60,40)
% .. controls (67.09875,61.00188) and (59.76253,84.57518) .. (40,90)
% .. controls (25.35715,94.01947) and (10.48064,84.5022) .. (10,70)
% .. controls (9.62895,58.80421) and (18.80421,49.62895) .. (30,50);
% 
% \end{tikzpicture}
% 
%
% \StopEventually{\PrintChanges}
% \section{Implementation}
%
% \subsection{Main Code}
%
% \iffalse
%<*hobby>
% \fi
%
% We use \LaTeX3 syntax so need to load the requisite packages
%    \begin{macrocode}
\RequirePackage{expl3}
\ExplSyntaxOn
%    \end{macrocode}
% 
% We declare all our variables.
% These are the globals.
%
% The function for computing the lengths of the control points depends on three parameters.
%    \begin{macrocode}
\fp_new:N \g_hobby_parama_fp
\fp_new:N \g_hobby_paramb_fp
\fp_new:N \g_hobby_paramc_fp

\fp_set:Nn \g_hobby_parama_fp {2}
\fp_pow:Nn \g_hobby_parama_fp {.5}

\fp_set:Nn \g_hobby_paramb_fp {1}
\fp_div:Nn \g_hobby_paramb_fp {16}

\fp_set:Nn \g_hobby_paramc_fp {5}
\fp_pow:Nn \g_hobby_paramc_fp {.5}
\fp_neg:N \g_hobby_paramc_fp
\fp_add:Nn \g_hobby_paramc_fp {3}
\fp_div:Nn \g_hobby_paramc_fp {2}
%    \end{macrocode}
%
% Now we define our objects for use in processing the path.
% \begin{macro}{\l_hobby_points_seq}
% \Verb+\l_hobby_points_seq+
%    \begin{macrocode}
\seq_new:N \l_hobby_points_seq
%    \end{macrocode}
% \end{macro}
%
% \begin{macro}{\l_hobby_path_tl}
% \Verb+\l_hobby_path_tl+
%    \begin{macrocode}
\tl_new:N \l_hobby_path_tl
%    \end{macrocode}
% \end{macro}
%
% \begin{macro}{\l_hobby_points_prop}
% \Verb+\l_hobby_points_prop+
%    \begin{macrocode}
\prop_new:N \l_hobby_points_prop
%    \end{macrocode}
% \end{macro}
%
% \begin{macro}{\l_hobby_pointsx_prop}
% \Verb+\l_hobby_pointsx_prop+
%    \begin{macrocode}
\prop_new:N \l_hobby_pointsx_prop
%    \end{macrocode}
% \end{macro}
%
% \begin{macro}{\l_hobby_pointsy_prop}
% \Verb+\l_hobby_pointsy_prop+
%    \begin{macrocode}
\prop_new:N \l_hobby_pointsy_prop
%    \end{macrocode}
% \end{macro}
%
% \begin{macro}{\l_hobby_angles_prop}
% \Verb+\l_hobby_angles_prop+
%    \begin{macrocode}
\prop_new:N \l_hobby_angles_prop
%    \end{macrocode}
% \end{macro}
%
% \begin{macro}{\l_hobby_distances_prop}
% \Verb+\l_hobby_distances_prop+
%    \begin{macrocode}
\prop_new:N \l_hobby_distances_prop
%    \end{macrocode}
% \end{macro}
%
% \begin{macro}{\l_hobby_tensionout_prop}
% \Verb+\l_hobby_tensionout_prop+
%    \begin{macrocode}
\prop_new:N \l_hobby_tensionout_prop
%    \end{macrocode}
% \end{macro}
%
% \begin{macro}{\l_hobby_tensionin_prop}
% \Verb+\l_hobby_tensionin_prop+
%    \begin{macrocode}
\prop_new:N \l_hobby_tensionin_prop
%    \end{macrocode}
% \end{macro}
%
% \begin{macro}{\l_hobby_tria_prop}
% \Verb+\l_hobby_tria_prop+
%    \begin{macrocode}
\prop_new:N \l_hobby_tria_prop
%    \end{macrocode}
% \end{macro}
%
% \begin{macro}{\l_hobby_trib_prop}
% \Verb+\l_hobby_trib_prop+
%    \begin{macrocode}
\prop_new:N \l_hobby_trib_prop
%    \end{macrocode}
% \end{macro}
%
% \begin{macro}{\l_hobby_tric_prop}
% \Verb+\l_hobby_tric_prop+
%    \begin{macrocode}
\prop_new:N \l_hobby_tric_prop
%    \end{macrocode}
% \end{macro}
%
% \begin{macro}{\l_hobby_trid_prop}
% \Verb+\l_hobby_trid_prop+
%    \begin{macrocode}
\prop_new:N \l_hobby_trid_prop
%    \end{macrocode}
% \end{macro}
%
% \begin{macro}{\l_hobby_psi_prop}
% \Verb+\l_hobby_psi_prop+
%    \begin{macrocode}
\prop_new:N \l_hobby_psi_prop
%    \end{macrocode}
% \end{macro}
%
% \begin{macro}{\l_hobby_psid_prop}
% \Verb+\l_hobby_psid_prop+
%    \begin{macrocode}
\prop_new:N \l_hobby_psid_prop
%    \end{macrocode}
% \end{macro}
%
% \begin{macro}{\l_hobby_theta_prop}
% \Verb+\l_hobby_theta_prop+
%    \begin{macrocode}
\prop_new:N \l_hobby_theta_prop
%    \end{macrocode}
% \end{macro}
%
% \begin{macro}{\l_hobby_phi_prop}
% \Verb+\l_hobby_phi_prop+
%    \begin{macrocode}
\prop_new:N \l_hobby_phi_prop
%    \end{macrocode}
% \end{macro}
%
% \begin{macro}{\l_hobby_sigma_prop}
% \Verb+\l_hobby_sigma_prop+
%    \begin{macrocode}
\prop_new:N \l_hobby_sigma_prop
%    \end{macrocode}
% \end{macro}
%
% \begin{macro}{\l_hobby_rho_prop}
% \Verb+\l_hobby_rho_prop+
%    \begin{macrocode}
\prop_new:N \l_hobby_rho_prop
%    \end{macrocode}
% \end{macro}
%
% \begin{macro}{\l_hobby_alpha_prop}
% \Verb+\l_hobby_alpha_prop+
%    \begin{macrocode}
\prop_new:N \l_hobby_alpha_prop
%    \end{macrocode}
% \end{macro}
%
% \begin{macro}{\l_hobby_controla_prop}
% \Verb+\l_hobby_controla_prop+
%    \begin{macrocode}
\prop_new:N \l_hobby_controla_prop
%    \end{macrocode}
% \end{macro}
%
% \begin{macro}{\l_hobby_controlb_prop}
% \Verb+\l_hobby_controlb_prop+
%    \begin{macrocode}
\prop_new:N \l_hobby_controlb_prop
%    \end{macrocode}
% \end{macro}
%
% \begin{macro}{\l_hobby_x_dim}
% \Verb+\l_hobby_x_dim+
%    \begin{macrocode}
\dim_new:N \l_hobby_x_dim
%    \end{macrocode}
% \end{macro}
%
% \begin{macro}{\l_hobby_y_dim}
% \Verb+\l_hobby_y_dim+
%    \begin{macrocode}
\dim_new:N \l_hobby_y_dim
%    \end{macrocode}
% \end{macro}
%
% \begin{macro}{\l_hobby_tempa_fp}
% \Verb+\l_hobby_tempa_fp+
%    \begin{macrocode}
\fp_new:N \l_hobby_tempa_fp
%    \end{macrocode}
% \end{macro}
%
% \begin{macro}{\l_hobby_tempb_fp}
% \Verb+\l_hobby_tempb_fp+
%    \begin{macrocode}
\fp_new:N \l_hobby_tempb_fp
%    \end{macrocode}
% \end{macro}
%
% \begin{macro}{\l_hobby_tempc_fp}
% \Verb+\l_hobby_tempc_fp+
%    \begin{macrocode}
\fp_new:N \l_hobby_tempc_fp
%    \end{macrocode}
% \end{macro}
%
% \begin{macro}{\l_hobby_tempd_fp}
% \Verb+\l_hobby_tempd_fp+
%    \begin{macrocode}
\fp_new:N \l_hobby_tempd_fp
%    \end{macrocode}
% \end{macro}
%
% \begin{macro}{\l_hobby_temps_fp}
% \Verb+\l_hobby_temps_fp+
%    \begin{macrocode}
\fp_new:N \l_hobby_temps_fp
%    \end{macrocode}
% \end{macro}
%
% \begin{macro}{\l_hobby_incurl_fp}
% \Verb+\l_hobby_incurl_fp+
%    \begin{macrocode}
\fp_new:N \l_hobby_incurl_fp
\fp_set:Nn \l_hobby_incurl_fp {1}
%    \end{macrocode}
% \end{macro}
%
% \begin{macro}{\l_hobby_outcurl_fp}
% \Verb+\l_hobby_outcurl_fp+
%    \begin{macrocode}
\fp_new:N \l_hobby_outcurl_fp
\fp_set:Nn \l_hobby_outcurl_fp {1}
%    \end{macrocode}
% \end{macro}
%
% \begin{macro}{\l_hobby_inang_fp}
% \Verb+\l_hobby_inang_fp+
%    \begin{macrocode}
\fp_new:N \l_hobby_inang_fp
\fp_set_eq:NN \l_hobby_inang_fp \c_undefined_fp
%    \end{macrocode}
% \end{macro}
%
% \begin{macro}{\l_hobby_outang_fp}
% \Verb+\l_hobby_outang_fp+
%    \begin{macrocode}
\fp_new:N \l_hobby_outang_fp
\fp_set_eq:NN \l_hobby_outang_fp \c_undefined_fp
%    \end{macrocode}
% \end{macro}
%
% \begin{macro}{\l_hobby_npoints_int}
% \Verb+\l_hobby_npoints_int+
%    \begin{macrocode}
\int_new:N \l_hobby_npoints_int
%    \end{macrocode}
% \end{macro}
%
% \begin{macro}{\l_hobby_temp_int}
% \Verb+\l_hobby_temp_int+
%    \begin{macrocode}
\int_new:N \l_hobby_temp_int
%    \end{macrocode}
% \end{macro}
%
% A ``point'' is a key-value list setting the x-value, the y-value, and the tensions at that point.
%    \begin{macrocode}
\keys_define:nn {hobby / read in all} {
  x .fp_set:N = \l_hobby_tempa_fp,
  y .fp_set:N = \l_hobby_tempb_fp,
  tension~out .fp_set:N = \l_hobby_tempc_fp,
  tension~in .fp_set:N = \l_hobby_tempd_fp,
  tension~out .default:n = 1,
  tension~in .default:n = 1,
}
\keys_define:nn { hobby / read in params} {
  in~angle .fp_set:N = \l_hobby_inang_fp,
  out~angle .fp_set:N = \l_hobby_outang_fp,
  in~curl .fp_set:N = \l_hobby_incurl_fp,
  out~curl .fp_set:N = \l_hobby_outcurl_fp,
}
%    \end{macrocode}
% \begin{macro}{\hobby_distangle:n}
% Computes the distance and angle between successive points.
% The argument given is the index of the current point.
% Assumptions: the points are in \Verb+\l_hobby_pointsx_prop+ and \Verb+\l_hobby_pointsy_prop+ and the index of the last point is \Verb+\l_hobby_npoints_int+.
%    \begin{macrocode}
\cs_set:Nn \hobby_distangle:n {
    \int_compare:nTF { #1 < \l_hobby_npoints_int }
    {
      \int_set:Nn \l_hobby_temp_int {#1}
      \int_incr:N \l_hobby_temp_int
    }
    {
      \int_seq:Nn \l_hobby_temp_int {0}
    }
    \prop_get:NnN \l_hobby_pointsx_prop {#1} \l_tmpa_tl
    \fp_set:Nn \l_hobby_tempa_fp {\l_tmpa_tl}
    \prop_get:NnN \l_hobby_pointsy_prop {#1} \l_tmpa_tl
    \fp_set:Nn \l_hobby_tempb_fp {\l_tmpa_tl}

    \exp_args:NNo \prop_get:NnN \l_hobby_pointsx_prop {\int_use:N \l_hobby_temp_int} \l_tmpa_tl
    \fp_sub:Nn \l_hobby_tempa_fp {\l_tmpa_tl}
    \exp_args:NNo \prop_get:NnN \l_hobby_pointsy_prop {\int_use:N \l_hobby_temp_int} \l_tmpa_tl
    \fp_sub:Nn \l_hobby_tempb_fp {\l_tmpa_tl}

    \fp_atantwo:NNN \l_hobby_tempc_fp \l_hobby_tempa_fp \l_hobby_tempb_fp
    \fp_veclen:NNN \l_hobby_tempd_fp \l_hobby_tempa_fp \l_hobby_tempb_fp

    \prop_put:Nnx \l_hobby_angles_prop {#1} {\fp_to_tl:N \l_hobby_tempc_fp}
    \prop_put:Nnx \l_hobby_distances_prop {#1} {\fp_to_tl:N \l_hobby_tempd_fp}
  }
%    \end{macrocode}
% \end{macro}
%
%
% \begin{macro}{\fp_atantwo:NNN}
% Computes the angle of the point specified by the latter two arguments, storing the answer in the first.
%    \begin{macrocode}
\cs_new:Nn \fp_atantwo:NNN {
  \pgfmathparse{rad(atan2(\fp_use:N #3,\fp_use:N #2))}
  \exp_args:NNo \fp_set:Nn #1 {\pgfmathresult}
}
%    \end{macrocode}
% \end{macro}
%
% \begin{macro}{\fp_veclen:NNN}
% Computes the length of the vector specified by the latter two arguments, storing the answer in the first.
% Needs temporary variables.
%    \begin{macrocode}
\fp_new:N \l_hobby_veclena_fp
\fp_new:N \l_hobby_veclenb_fp
\cs_new:Nn \fp_veclen:NNN {
  \fp_set_eq:NN \l_hobby_veclena_fp #2
  \fp_set_eq:NN \l_hobby_veclenb_fp #3
  \fp_mul:Nn \l_hobby_veclena_fp {\l_hobby_veclena_fp}
  \fp_mul:Nn \l_hobby_veclenb_fp {\l_hobby_veclenb_fp}
  \fp_add:Nn \l_hobby_veclena_fp {\l_hobby_veclenb_fp}
  \fp_pow:Nn \l_hobby_veclena_fp {.5}
  \fp_set_eq:NN #1 \l_hobby_veclena_fp
}
%    \end{macrocode}
% \end{macro}
%
% \begin{macro}{\hobby_genpath:}
% This is the curve generation function.
% We assume at the start that we have a property list containing all the points that the curve must go through, and the various curve parameters have been initialised.
% So these must be set up by a wrapper function which then calls this one.
% The list of required information is:
% \begin{enumerate}
% \item \Verb+\l_hobby_pointsx_prop+
% \item \Verb+\l_hobby_pointsy_prop+
% \item \Verb+\l_hobby_npoints_int+
% \item \Verb+\l_hobby_tensionout_prop+
% \item \Verb+\l_hobby_tensionin_prop+
% \item \Verb+\l_hobby_incurl_fp+
% \item \Verb+\l_hobby_outcurl_fp+
% \item \Verb+\l_hobby_inang_fp+
% \item \Verb+\l_hobby_outang_fp+
% \end{enumerate}
%
%    \begin{macrocode}
\cs_new:Nn \hobby_genpath:
{
%    \end{macrocode}
% Our first step is to go through the list of points and compute the distances and angles between successive points.
% For an open path, we iterate through the points from \(0\) to \(n-1\).
% Thus \(d_i\) is the distance from \(z_i\) to \(z_{i+1}\) and the angle is the angle of the line from \(z_i\) to \(z_{i+1}\).

%    \begin{macrocode}
\prg_stepwise_function:nnnN {0} {1} {\l_hobby_npoints_int - 1} \hobby_distangle:n
%    \end{macrocode}
%
% For the majority of the code, we're only really interested in the differences of the angles.
% So for each internal point we compute the differences in the angles.
%    \begin{macrocode}
  \prg_stepwise_inline:nnnn {1} {1} {\l_hobby_npoints_int - 1} {
    \exp_args:NNx \prop_get:NnN \l_hobby_angles_prop {\int_eval:n {##1 - 1}} \l_tmpa_tl
    \fp_set:Nn \l_hobby_tempa_fp {\tl_use:N \l_tmpa_tl}
    \prop_get:NnN \l_hobby_angles_prop {##1} \l_tmpa_tl
    \fp_sub:Nn \l_hobby_tempa_fp {\tl_use:N \l_tmpa_tl}
    \fp_compare:nNnTF {\l_hobby_tempa_fp} > { \c_pi_fp }
    {
      \fp_sub:Nn \l_hobby_tempa_fp {\c_pi_fp}
      \fp_sub:Nn \l_hobby_tempa_fp {\c_pi_fp}
    }
    {}
    \fp_set_eq:NN \l_hobby_tempb_fp \l_hobby_tempa_fp
    \fp_neg:N \l_hobby_tempb_fp
    \fp_compare:nNnTF {\l_hobby_tempb_fp} > {\c_pi_fp }
    {
      \fp_add:Nn \l_hobby_tempa_fp {\c_pi_fp}
      \fp_add:Nn \l_hobby_tempa_fp {\c_pi_fp}
    }
    {}
    \prop_put:Nnx \l_hobby_psi_prop {##1}{\fp_to_tl:N \l_hobby_tempa_fp}
  }
%    \end{macrocode}
%
% Next, we generate the matrix.
% We start with the subdiagonal.
% This is indexed from \(1\) to \(n-1\).
%    \begin{macrocode}
  \prg_stepwise_inline:nnnn {1} {1} {\l_hobby_npoints_int - 1} {
    \prop_get:NnN \l_hobby_tensionin_prop {##1} \l_tmpa_tl
    \fp_set:Nn \l_hobby_tempa_fp {\l_tmpa_tl}
    \fp_mul:Nn \l_hobby_tempa_fp {\l_tmpa_tl}
    \prop_get:NnN \l_hobby_distances_prop {##1} \l_tmpa_tl
    \fp_mul:Nn \l_hobby_tempa_fp {\l_tmpa_tl}
    \exp_args:NNx \prop_get:NnN \l_hobby_tensionin_prop {\int_eval:n {##1 + 1}} \l_tmpa_tl
    \fp_mul:Nn \l_hobby_tempa_fp {\l_tmpa_tl}
    \prop_put:Nnx \l_hobby_tria_prop {##1} {\fp_to_tl:N \l_hobby_tempa_fp}
  }
%    \end{macrocode}
%
% Next, we attack main diagonal.
% We might need to adjust the first and last terms, but we'll do that in a minute.
%    \begin{macrocode}
  \prg_stepwise_inline:nnnn {1} {1} {\l_hobby_npoints_int - 1} {

  \exp_args:NNx \prop_get:NnN \l_hobby_tensionin_prop {\int_eval:n {##1 + 1}} \l_tmpa_tl
  \fp_set:Nn \l_hobby_tempa_fp {\l_tmpa_tl}
  \fp_mul:Nn \l_hobby_tempa_fp {3}
  \fp_sub:Nn \l_hobby_tempa_fp {1}

  \prop_get:NnN \l_hobby_tensionout_prop {##1} \l_tmpa_tl
  \fp_mul:Nn \l_hobby_tempa_fp {\l_tmpa_tl}
  \fp_mul:Nn \l_hobby_tempa_fp {\l_tmpa_tl}

  \exp_args:NNx \prop_get:NnN \l_hobby_tensionout_prop {\int_eval:n {##1 - 1}} \l_tmpa_tl
  \fp_mul:Nn \l_hobby_tempa_fp {\l_tmpa_tl}

  \exp_args:NNx \prop_get:NnN \l_hobby_distances_prop {\int_eval:n {##1 - 1}} \l_tmpa_tl
  \fp_mul:Nn \l_hobby_tempa_fp {\l_tmpa_tl}

  \exp_args:NNx \prop_get:NnN \l_hobby_tensionout_prop {\int_eval:n {##1 - 1}} \l_tmpa_tl
  \fp_set:Nn \l_hobby_tempb_fp {\l_tmpa_tl}
  \fp_mul:Nn \l_hobby_tempb_fp {3}
  \fp_sub:Nn \l_hobby_tempb_fp {1}

  \prop_get:NnN \l_hobby_tensionin_prop {##1} \l_tmpa_tl
  \fp_mul:Nn \l_hobby_tempb_fp {\l_tmpa_tl}
  \fp_mul:Nn \l_hobby_tempb_fp {\l_tmpa_tl}

  \exp_args:NNx \prop_get:NnN \l_hobby_tensionin_prop {\int_eval:n {##1 + 1}} \l_tmpa_tl
  \fp_mul:Nn \l_hobby_tempb_fp {\l_tmpa_tl}

  \prop_get:NnN \l_hobby_distances_prop {##1} \l_tmpa_tl
  \fp_mul:Nn \l_hobby_tempb_fp {\l_tmpa_tl}

  \fp_add:Nn \l_hobby_tempa_fp {\l_hobby_tempb_fp}

  \prop_put:Nnx \l_hobby_trib_prop {##1} {\fp_to_tl:N \l_hobby_tempa_fp}
}
%    \end{macrocode}
%
% Next, the superdiagonal.
%    \begin{macrocode}
  \prg_stepwise_inline:nnnn {1} {1} {\l_hobby_npoints_int - 2} {
  \prop_get:NnN \l_hobby_tensionin_prop {##1} \l_tmpa_tl
  \fp_set:Nn \l_hobby_tempa_fp {\l_tmpa_tl}
  \fp_mul:Nn \l_hobby_tempa_fp {\l_tmpa_tl}

  \exp_args:NNx \prop_get:NnN \l_hobby_tensionin_prop {\int_eval:n {##1 - 1}} \l_tmpa_tl
  \fp_mul:Nn \l_hobby_tempa_fp {\l_tmpa_tl}

  \exp_args:NNx \prop_get:NnN \l_hobby_distances_prop {\int_eval:n {##1 - 1}} \l_tmpa_tl
  \fp_mul:Nn \l_hobby_tempa_fp {\l_tmpa_tl}

  \prop_put:Nnx \l_hobby_tric_prop {##1} {\fp_to_tl:N \l_hobby_tempa_fp}
}
%    \end{macrocode}
%
% Lastly (before the adjustments), the target vector.
%    \begin{macrocode}
  \prg_stepwise_inline:nnnn {1} {1} {\l_hobby_npoints_int - 2} {
  \exp_args:NNx \prop_get:NnN \l_hobby_psi_prop {\int_eval:n {##1 + 1}} \l_tmpa_tl
  \fp_set:Nn \l_hobby_tempa_fp {\l_tmpa_tl}

  \prop_get:NnN \l_hobby_tensionout_prop {##1} \l_tmpa_tl
  \fp_mul:Nn \l_hobby_tempa_fp {\l_tmpa_tl}
  \fp_mul:Nn \l_hobby_tempa_fp {\l_tmpa_tl}

  \exp_args:NNx \prop_get:NnN \l_hobby_tensionout_prop {\int_eval:n {##1 - 1}} \l_tmpa_tl
  \fp_mul:Nn \l_hobby_tempa_fp {\l_tmpa_tl}

  \exp_args:NNx \prop_get:NnN \l_hobby_distances_prop {\int_eval:n {##1 - 1}} \l_tmpa_tl
  \fp_mul:Nn \l_hobby_tempa_fp {\l_tmpa_tl}

  \fp_neg:N \l_hobby_tempa_fp

  \exp_args:NNx \prop_get:NnN \l_hobby_tensionout_prop {\int_eval:n {##1 - 1}} \l_tmpa_tl
  \fp_set:Nn \l_hobby_tempb_fp {\l_tmpa_tl}
  \fp_mul:Nn \l_hobby_tempb_fp {3}
  \fp_sub:Nn \l_hobby_tempb_fp {1}

  \prop_get:NnN \l_hobby_psi_prop {##1} \l_tmpa_tl
  \fp_mul:Nn \l_hobby_tempb_fp {\l_tmpa_tl}

  \prop_get:NnN \l_hobby_tensionin_prop {##1} \l_tmpa_tl
  \fp_mul:Nn \l_hobby_tempb_fp {\l_tmpa_tl}
  \fp_mul:Nn \l_hobby_tempb_fp {\l_tmpa_tl}

  \exp_args:NNx \prop_get:NnN \l_hobby_tensionin_prop {\int_eval:n {##1 + 1}} \l_tmpa_tl
  \fp_mul:Nn \l_hobby_tempb_fp {\l_tmpa_tl}

  \prop_get:NnN \l_hobby_distances_prop {##1} \l_tmpa_tl
  \fp_mul:Nn \l_hobby_tempb_fp {\l_tmpa_tl}

  \fp_sub:Nn \l_hobby_tempa_fp {\l_hobby_tempb_fp}

  \prop_put:Nnx \l_hobby_trid_prop {##1} {\fp_to_tl:N \l_hobby_tempa_fp}
}
%    \end{macrocode}
%
% Next, there are some adjustments at the ends.
% First, we test to see if \(\theta_0\) has been specified.
%    \begin{macrocode}
\fp_if_undefined:NTF \l_hobby_inang_fp
{
  \prop_get:NnN \l_hobby_tensionin_prop {1} \l_tmpa_tl
  \fp_set:Nn \l_hobby_tempa_fp {\l_tmpa_tl}
  \fp_mul:Nn \l_hobby_tempa_fp {\l_tmpa_tl}
  \fp_mul:Nn \l_hobby_tempa_fp {\l_tmpa_tl}
  \fp_mul:Nn \l_hobby_tempa_fp {\l_hobby_incurl_fp}

  \prop_get:NnN \l_hobby_tensionin_prop {1} \l_tmpa_tl
  \fp_set:Nn \l_hobby_tempb_fp {\l_tmpa_tl}
  \fp_mul:Nn \l_hobby_tempb_fp {3}
  \fp_sub:Nn \l_hobby_tempb_fp {1}

  \prop_get:NnN \l_hobby_tensionout_prop {0} \l_tmpa_tl
  \fp_mul:Nn \l_hobby_tempb_fp {\l_tmpa_tl}
  \fp_mul:Nn \l_hobby_tempb_fp {\l_tmpa_tl}
  \fp_mul:Nn \l_hobby_tempb_fp {\l_tmpa_tl}

  \fp_add:Nn \l_hobby_tempa_fp {\l_hobby_tempb_fp}
  
  \prop_put:Nnx \l_hobby_trib_prop {0}  {\fp_to_tl:N \l_hobby_tempa_fp}

  \prop_get:NnN \l_hobby_tensionout_prop {0} \l_tmpa_tl
  \fp_set:Nn \l_hobby_tempa_fp {\l_tmpa_tl}
  \fp_mul:Nn \l_hobby_tempa_fp {\l_tmpa_tl}
  \fp_mul:Nn \l_hobby_tempa_fp {\l_tmpa_tl}

  \prop_get:NnN \l_hobby_tensionout_prop {0} \l_tmpa_tl
  \fp_set:Nn \l_hobby_tempb_fp {\l_tmpa_tl}
  \fp_mul:Nn \l_hobby_tempb_fp {3}
  \fp_sub:Nn \l_hobby_tempb_fp {1}

  \prop_get:NnN \l_hobby_tensionin_prop {1} \l_tmpa_tl
  \fp_mul:Nn \l_hobby_tempb_fp {\l_tmpa_tl}
  \fp_mul:Nn \l_hobby_tempb_fp {\l_tmpa_tl}
  \fp_mul:Nn \l_hobby_tempb_fp {\l_tmpa_tl}

  \fp_mul:Nn \l_hobby_tempb_fp {\l_hobby_incurl_fp}

  \fp_add:Nn \l_hobby_tempa_fp {\l_hobby_tempb_fp}
  
  \prop_put:Nnx \l_hobby_tric_prop {0} {\fp_to_tl:N \l_hobby_tempa_fp}

  \fp_neg:N \l_hobby_tempa_fp

  \prop_get:NnN \l_hobby_psi_prop {1} \l_tmpa_tl
  \fp_mul:Nn \l_hobby_tempa_fp {\l_tmpa_tl}
  
  \prop_put:Nnx \l_hobby_trid_prop {0} {\fp_to_tl:N \l_hobby_tempa_fp}
  
}
{
  \prop_put:Nnn \l_hobby_trib_prop {0} {1}
  \prop_put:Nnn \l_hobby_tric_prop {0} {0}
  \prop_put:Nnx \l_hobby_trid_prop {0} {\fp_to_tl:N \l_hobby_inang_fp}
}
%    \end{macrocode}
%
% Next, if \(\psi_n\) has been given.
%    \begin{macrocode}
\fp_if_undefined:NTF \l_hobby_outang_fp
{
  \exp_args:NNx \prop_get:NnN \l_hobby_tensionout_prop {\int_eval:n {\l_hobby_npoints_int - 1}} \l_tmpa_tl
  \fp_set:Nn \l_hobby_tempa_fp {\l_tmpa_tl}
  \fp_mul:Nn \l_hobby_tempa_fp {\l_tmpa_tl}

  \exp_args:NNx \prop_get:NnN \l_hobby_tensionout_prop {\int_eval:n {\l_hobby_npoints_int - 2}} \l_tmpa_tl
  \fp_mul:Nn \l_hobby_tempa_fp {\l_tmpa_tl}

  \exp_args:NNx \prop_get:NnN \l_hobby_distances_prop {\int_eval:n {\l_hobby_npoints_int - 2}} \l_tmpa_tl
  \fp_mul:Nn \l_hobby_tempa_fp {\l_tmpa_tl}

  \exp_args:NNo \prop_get:NnN \l_hobby_tensionin_prop {\int_use:N \l_hobby_npoints_int} \l_tmpa_tl
  \fp_set:Nn \l_hobby_tempb_fp {\l_tmpa_tl}
  \fp_mul:Nn \l_hobby_tempb_fp {3}
  \fp_sub:Nn \l_hobby_tempb_fp {1}

  \exp_args:NNx \prop_get:NnN \l_hobby_tensionout_prop {\int_eval:n     {\l_hobby_npoints_int - 1}} \l_tmpa_tl
  \fp_mul:Nn \l_hobby_tempb_fp {\l_tmpa_tl}
  \fp_mul:Nn \l_hobby_tempb_fp {\l_tmpa_tl}
  \fp_mul:Nn \l_hobby_tempb_fp {\l_tmpa_tl}

  \fp_mul:Nn \l_hobby_tempb_fp {\l_hobby_outcurl_fp}

  \exp_args:NNo \prop_get:NnN \l_hobby_tensionin_prop {\int_use:N \l_hobby_npoints_int } \l_tmpa_tl
  \fp_set:Nn \l_hobby_tempc_fp {\l_tmpa_tl}
  \fp_mul:Nn \l_hobby_tempc_fp {\l_tmpa_tl}
  \fp_mul:Nn \l_hobby_tempc_fp {\l_tmpa_tl}

  \fp_add:Nn \l_hobby_tempb_fp {\l_hobby_tempc_fp}

  \fp_mul:Nn \l_hobby_tempa_fp {\l_hobby_tempb_fp}

  \exp_args:NNx \prop_get:NnN \l_hobby_tensionout_prop {\int_eval:n {\l_hobby_npoints_int -2}} \l_tmpa_tl
  \fp_set:Nn \l_hobby_tempb_fp {\l_tmpa_tl}
  \fp_mul:Nn \l_hobby_tempb_fp {3}
  \fp_sub:Nn \l_hobby_tempb_fp {1}

  \exp_args:NNx \prop_get:NnN \l_hobby_tensionin_prop {\int_use:N \l_hobby_npoints_int} \l_tmpa_tl
  \fp_mul:Nn \l_hobby_tempb_fp {\l_tmpa_tl}
  \fp_mul:Nn \l_hobby_tempb_fp {\l_tmpa_tl}
  \fp_mul:Nn \l_hobby_tempb_fp {\l_tmpa_tl}

  \exp_args:NNx \prop_get:NnN \l_hobby_tensionout_prop {\int_eval:n {\l_hobby_npoints_int - 1}} \l_tmpa_tl
  \fp_set:Nn \l_hobby_tempc_fp {\l_tmpa_tl}
  \fp_mul:Nn \l_hobby_tempc_fp {\l_tmpa_tl}
  \fp_mul:Nn \l_hobby_tempc_fp {\l_tmpa_tl}

  \fp_mul:Nn \l_hobby_tempc_fp {\l_hobby_outcurl_fp}

  \fp_add:Nn \l_hobby_tempb_fp {\l_hobby_tempc_fp}

  \fp_div:Nn \l_hobby_tempa_fp {\l_hobby_tempb_fp}

  \fp_neg:N \l_hobby_tempa_fp

  \exp_args:NNx \prop_get:NnN \l_hobby_trib_prop {\int_eval:n {\l_hobby_npoints_int - 1}} \l_tmpa_tl

  \fp_add:Nn \l_hobby_tempa_fp {\l_tmpa_tl}

  \exp_args:NNx \prop_put:Nnx \l_hobby_trib_prop {\int_eval:n {\l_hobby_npoints_int - 1}} {\fp_to_tl:N \l_hobby_tempa_fp}


  \exp_args:NNx \prop_get:NnN \l_hobby_tensionout_prop {\int_eval:n {\l_hobby_npoints_int - 2}} \l_tmpa_tl
  \fp_set:Nn \l_hobby_tempb_fp {\l_tmpa_tl}
  \fp_mul:Nn \l_hobby_tempb_fp {3}
  \fp_sub:Nn \l_hobby_tempb_fp {1}

  \exp_args:NNx \prop_get:NnN \l_hobby_psi_prop {\int_eval:n {\l_hobby_npoints_int - 1}} \l_tmpa_tl
  \fp_mul:Nn \l_hobby_tempb_fp {\l_tmpa_tl}

  \exp_args:NNx \prop_get:NnN \l_hobby_tensionin_prop {\int_eval:n {\l_hobby_npoints_int - 1}} \l_tmpa_tl
  \fp_mul:Nn \l_hobby_tempb_fp {\l_tmpa_tl}
  \fp_mul:Nn \l_hobby_tempb_fp {\l_tmpa_tl}

  \exp_args:NNo \prop_get:NnN \l_hobby_tensionin_prop {\int_use:N \l_hobby_npoints_int} \l_tmpa_tl
  \fp_mul:Nn \l_hobby_tempb_fp {\l_tmpa_tl}

  \exp_args:NNx \prop_get:NnN \l_hobby_distances_prop {\int_eval:n {\l_hobby_npoints_int - 1}} \l_tmpa_tl
  \fp_mul:Nn \l_hobby_tempb_fp {\l_tmpa_tl}

  \fp_neg:N \l_hobby_tempb_fp

  \exp_args:NNx \prop_put:Nnx \l_hobby_trid_prop {\int_eval:n {\l_hobby_npoints_int - 1}} {\fp_to_tl:N \l_hobby_tempb_fp}
}
{
  \fp_set_eq:NN \l_hobby_tempa_fp \l_hobby_outang_fp

  \exp_args:NNx \prop_get:NnN \l_hobby_tensionout_prop {\int_eval:n {\l_hobby_npoints_int - 1}} \l_tmpa_tl
  \fp_mul:Nn \l_hobby_tempa_fp {\l_tmpa_tl}
  \fp_mul:Nn \l_hobby_tempa_fp {\l_tmpa_tl}

  \exp_args:NNx \prop_get:NnN \l_hobby_tensionout_prop {\int_eval:n {\l_hobby_npoints_in - 2}} \l_tmpa_tl
  \fp_mul:Nn \l_hobby_tempa_fp {\l_tmpa_tl}

  \exp_args:NNx \prop_get:NnN \l_hobby_distances_prop {\int_eval:n {\l_hobby_npoints_int - 2}} \l_tmpa_tl
  \fp_mul:Nn \l_hobby_tempa_fp {\l_tmpa_tl}

  \fp_neg:N \l_hobby_tempa_fp

  \exp_args:NNx \prop_get:NnN \l_hobby_tensionout_prop {\int_eval:n {\l_hobby_npoints_int - 2}} \l_tmpa_tl
  \fp_set:Nn \l_hobby_tempb_fp {\l_tmpa_tl}
  \fp_mul:Nn \l_hobby_tempb_fp {3}
  \fp_sub:Nn \l_hobby_tempb_fp {1}

  \exp_args:NNx \prop_get:NnN \l_hobby_psi_prop {\int_eval:n {\l_hobby_npoints_int - 1}} \l_tmpa_tl
  \fp_mul:Nn \l_hobby_tempb_fp {\l_tmpa_tl}

  \exp_args:NNx \prop_get:NnN \l_hobby_tensionin_prop {\int_eval:n {\l_hobby_npoints_int - 1}} \l_tmpa_tl
  \fp_mul:Nn \l_hobby_tempb_fp {\l_tmpa_tl}
  \fp_mul:Nn \l_hobby_tempb_fp {\l_tmpa_tl}

  \exp_args:NNo \prop_get:NnN \l_hobby_tensionin_prop {\int_use:N \l_hobby_npoints_int} \l_tmpa_tl
  \fp_mul:Nn \l_hobby_tempb_fp {\l_tmpa_tl}

  \exp_args:NNx \prop_get:NnN \l_hobby_distances_prop {\int_eval:n {\l_hobby_npoints_int - 1}} \l_tmpa_tl
  \fp_mul:Nn \l_hobby_tempb_fp {\l_tmpa_tl}

  \fp_sub:Nn \l_hobby_tempa_fp {\l_hobby_tempb_fp}

  \exp_args:NNx \prop_put:Nnx \l_hobby_trid_prop {\int_eval:n {\l_hobby_npoints_int - 1}} {\fp_to_tl:N \l_hobby_tempa_fp}
}
%    \end{macrocode}
%
% Now we have the tridiagonal matrix in place, we implement the solution.
%
%    \begin{macrocode}
\prg_stepwise_inline:nnnn {1} {1} {\l_hobby_npoints_int - 1} {
  \exp_args:NNx \prop_get:NnN \l_hobby_trib_prop {\int_eval:n{##1 - 1}} \l_tmpa_tl
  \fp_set:Nn \l_hobby_tempb_fp {\l_tmpa_tl}
  \prop_get:NnN \l_hobby_trib_prop {##1} \l_tmpa_tl
  \fp_mul:Nn \l_hobby_tempb_fp {\l_tmpa_tl}

  \exp_args:NNx \prop_get:NnN \l_hobby_tric_prop {\int_eval:n{##1 - 1}} \l_tmpa_tl
  \fp_set:Nn \l_hobby_tempa_fp {\l_tmpa_tl}
  \prop_get:NnN \l_hobby_tria_prop {##1} \l_tmpa_tl
  \fp_mul:Nn \l_hobby_tempa_fp {\l_tmpa_tl}

  \fp_sub:Nn \l_hobby_tempb_fp {\l_hobby_tempa_fp}

  \prop_put:Nnx \l_hobby_trib_prop {##1} {\fp_to_tl:N \l_hobby_tempb_fp}

  \int_compare:nTF {##1 < \l_hobby_npoints_int - 1} {

  \exp_args:NNx \prop_get:NnN \l_hobby_trib_prop {\int_eval:n{##1 - 1}} \l_tmpa_tl
  \fp_set:Nn \l_hobby_tempa_fp {\l_tmpa_tl}
  \prop_get:NnN \l_hobby_tric_prop {##1} \l_tmpa_tl
  \fp_mul:Nn \l_hobby_tempa_fp {\l_tmpa_tl}

  \prop_put:Nnx \l_hobby_tric_prop {##1} {\fp_to_tl:N \l_hobby_tempa_fp}
  }
  {}

  \exp_args:NNx \prop_get:NnN \l_hobby_trib_prop {\int_eval:n{##1 - 1}} \l_tmpa_tl
  \fp_set:Nn \l_hobby_tempb_fp {\l_tmpa_tl}
  \prop_get:NnN \l_hobby_trid_prop {##1} \l_tmpa_tl
  \fp_mul:Nn \l_hobby_tempb_fp {\l_tmpa_tl}

  \exp_args:NNx \prop_get:NnN \l_hobby_trid_prop {\int_eval:n{##1 - 1}} \l_tmpa_tl
  \fp_set:Nn \l_hobby_tempa_fp {\l_tmpa_tl}
  \prop_get:NnN \l_hobby_tria_prop {##1} \l_tmpa_tl
  \fp_mul:Nn \l_hobby_tempa_fp {\l_tmpa_tl}

  \fp_sub:Nn \l_hobby_tempb_fp {\l_hobby_tempa_fp}

  \prop_put:Nnx \l_hobby_trid_prop {##1} {\fp_to_tl:N \l_hobby_tempb_fp}
}

\exp_args:NNx \prop_get:NnN \l_hobby_trid_prop {\int_eval:n {\l_hobby_npoints_int - 1}} \l_tmpa_tl
\fp_set:Nn \l_hobby_tempa_fp {\l_tmpa_tl}

\exp_args:NNx \prop_get:NnN \l_hobby_trib_prop {\int_eval:n {\l_hobby_npoints_int - 1}} \l_tmpa_tl
\fp_div:Nn \l_hobby_tempa_fp {\l_tmpa_tl}

\exp_args:NNx \prop_put:Nnx \l_hobby_theta_prop {\int_eval:n {\l_hobby_npoints_int - 1}} {\fp_to_tl:N \l_hobby_tempa_fp}

\prg_stepwise_inline:nnnn {\l_hobby_npoints_int - 2} {-1} {0} {
  \exp_args:NNx \prop_get:NnN \l_hobby_theta_prop {\int_eval:n {##1 + 1}} \l_tmpa_tl
  \fp_set:Nn \l_hobby_tempa_fp {\l_tmpa_tl}
  \prop_get:NnN \l_hobby_tric_prop {##1} \l_tmpa_tl
  \fp_mul:Nn \l_hobby_tempa_fp {\l_tmpa_tl}
  \prop_get:NnN \l_hobby_trid_prop {##1} \l_tmpa_tl
  \fp_neg:N \l_hobby_tempa_fp
  \fp_add:Nn \l_hobby_tempa_fp {\l_tmpa_tl}
  \prop_get:NnN \l_hobby_trib_prop {##1} \l_tmpa_tl
  \fp_div:Nn \l_hobby_tempa_fp {\l_tmpa_tl}
  \prop_put:Nnx \l_hobby_theta_prop {##1} {\fp_to_tl:N \l_hobby_tempa_fp}
}
%    \end{macrocode}
%
% Now that we have computed the \(\theta_i\)s, we can quickly compute the \(\phi_i\)s.
%
%    \begin{macrocode}
\prg_stepwise_inline:nnnn {1} {1} {\l_hobby_npoints_int - 1} {
    \prop_get:NnN \l_hobby_theta_prop {##1} \l_tmpa_tl
    \fp_set:Nn \l_hobby_tempa_fp {\l_tmpa_tl}
    \prop_get:NnN \l_hobby_psi_prop {##1} \l_tmpa_tl
    \fp_add:Nn \l_hobby_tempa_fp {\l_tmpa_tl}
    \fp_neg:N \l_hobby_tempa_fp
    \prop_put:Nnx \l_hobby_phi_prop {##1} {\fp_to_tl:N \l_hobby_tempa_fp}
  }
%    \end{macrocode}
%
% This works for all except \(\psi_n\).
%
%    \begin{macrocode}
  \exp_args:NNo \prop_get:NnN \l_hobby_tensionin_prop {\int_use:N \l_hobby_npoints_int} \l_tmpa_tl
  \fp_set:Nn \l_hobby_tempa_fp {\l_tmpa_tl}
  \fp_mul:Nn \l_hobby_tempa_fp {3}
  \fp_sub:Nn \l_hobby_tempa_fp {1}

  \exp_args:NNx \prop_get:NnN \l_hobby_tensionout_prop {\int_eval:n     {\l_hobby_npoints_int - 1}} \l_tmpa_tl
  \fp_mul:Nn \l_hobby_tempa_fp {\l_tmpa_tl}
  \fp_mul:Nn \l_hobby_tempa_fp {\l_tmpa_tl}
  \fp_mul:Nn \l_hobby_tempa_fp {\l_tmpa_tl}

  \fp_mul:Nn \l_hobby_tempa_fp {\l_hobby_outcurl_fp}

  \exp_args:NNo \prop_get:NnN \l_hobby_tensionin_prop {\int_use:N \l_hobby_npoints_int } \l_tmpa_tl
  \fp_set:Nn \l_hobby_tempb_fp {\l_tmpa_tl}
  \fp_mul:Nn \l_hobby_tempb_fp {\l_tmpa_tl}
  \fp_mul:Nn \l_hobby_tempb_fp {\l_tmpa_tl}

  \fp_add:Nn \l_hobby_tempa_fp {\l_hobby_tempb_fp}

  \exp_args:NNx \prop_get:NnN \l_hobby_tensionout_prop {\int_eval:n {\l_hobby_npoints_int -2}} \l_tmpa_tl
  \fp_set:Nn \l_hobby_tempb_fp {\l_tmpa_tl}
  \fp_mul:Nn \l_hobby_tempb_fp {3}
  \fp_sub:Nn \l_hobby_tempb_fp {1}

  \exp_args:NNx \prop_get:NnN \l_hobby_tensionin_prop {\int_use:N \l_hobby_npoints_int} \l_tmpa_tl
  \fp_mul:Nn \l_hobby_tempb_fp {\l_tmpa_tl}
  \fp_mul:Nn \l_hobby_tempb_fp {\l_tmpa_tl}
  \fp_mul:Nn \l_hobby_tempb_fp {\l_tmpa_tl}

  \exp_args:NNx \prop_get:NnN \l_hobby_tensionout_prop {\int_eval:n {\l_hobby_npoints_int - 1}} \l_tmpa_tl
  \fp_set:Nn \l_hobby_tempc_fp {\l_tmpa_tl}
  \fp_mul:Nn \l_hobby_tempc_fp {\l_tmpa_tl}
  \fp_mul:Nn \l_hobby_tempc_fp {\l_tmpa_tl}

  \fp_mul:Nn \l_hobby_tempc_fp {\l_hobby_outcurl_fp}

  \fp_add:Nn \l_hobby_tempb_fp {\l_hobby_tempc_fp}

  \fp_div:Nn \l_hobby_tempa_fp {\l_hobby_tempb_fp}

\exp_args:NNx \prop_get:NnN \l_hobby_theta_prop {\int_eval:n {\l_hobby_npoints_int -1}} \l_tmpa_tl
\fp_mul:Nn \l_hobby_tempa_fp {\l_tmpa_tl}

\exp_args:NNx \prop_put:Nnx \l_hobby_phi_prop {\int_use:N \l_hobby_npoints_int} {\fp_to_tl:N \l_hobby_tempa_fp}
%    \end{macrocode}
%
% Next task is to compute the \(\rho_i\) and \(\sigma_i\).
%    \begin{macrocode}
\prg_stepwise_inline:nnnn {0} {1} {\l_hobby_npoints_int - 1} {

  \prop_get:NnN \l_hobby_theta_prop {##1} \l_tmpa_tl
  \fp_set:Nn \l_hobby_tempa_fp {\l_tmpa_tl}

    \exp_args:NNx \prop_get:NnN \l_hobby_phi_prop {\int_eval:n {##1 + 1}} \l_tmpa_tl
  
    \fp_set:Nn \l_hobby_tempb_fp {\l_tmpa_tl}

    \fp_set:Nn \l_hobby_temps_fp {0}

    \fp_cos:Nn \l_hobby_tempc_fp {\l_hobby_tempa_fp}
    \fp_cos:Nn \l_hobby_tempd_fp {\l_hobby_tempb_fp}

    \fp_set_eq:NN \l_hobby_temps_fp \l_hobby_tempc_fp
    \fp_sub:Nn \l_hobby_temps_fp {\l_hobby_tempd_fp}

    \fp_sin:Nn \l_hobby_tempc_fp {\l_hobby_tempa_fp}
    \fp_sin:Nn \l_hobby_tempd_fp {\l_hobby_tempb_fp}

    \fp_mul:Nn \l_hobby_tempc_fp {\g_hobby_paramb_fp}
    \fp_sub:Nn \l_hobby_tempd_fp {\l_hobby_tempc_fp}
    \fp_mul:Nn \l_hobby_temps_fp {\l_hobby_tempd_fp}

    \fp_sin:Nn \l_hobby_tempc_fp {\l_hobby_tempa_fp}
    \fp_sin:Nn \l_hobby_tempd_fp {\l_hobby_tempb_fp}

    \fp_mul:Nn \l_hobby_tempd_fp {\g_hobby_paramb_fp}
    \fp_sub:Nn \l_hobby_tempc_fp {\l_hobby_tempd_fp}
    \fp_mul:Nn \l_hobby_temps_fp {\l_hobby_tempc_fp}

    \fp_mul:Nn \l_hobby_temps_fp {\g_hobby_parama_fp}

    \fp_cos:Nn \l_hobby_tempa_fp {\l_hobby_tempa_fp}
    \fp_cos:Nn \l_hobby_tempb_fp {\l_hobby_tempb_fp}

    \fp_set_eq:NN \l_hobby_tempc_fp \l_hobby_tempa_fp
    \fp_sub:Nn \l_hobby_tempc_fp {\l_hobby_tempb_fp}
    \fp_mul:Nn \l_hobby_tempc_fp {\g_hobby_paramc_fp}

    \fp_add:Nn \l_hobby_tempc_fp {\l_hobby_tempb_fp}
    \fp_add:Nn \l_hobby_tempc_fp {1}

    \fp_set_eq:NN \l_hobby_tempd_fp \l_hobby_temps_fp

    \fp_neg:N \l_hobby_tempd_fp
    \fp_add:Nn \l_hobby_tempd_fp {2}

    \fp_div:Nn \l_hobby_tempd_fp {\l_hobby_tempc_fp}

  \exp_args:NNx \prop_put:Nnx \l_hobby_sigma_prop {\int_eval:n{##1 + 1}} {\fp_to_tl:N \l_hobby_tempd_fp}

    \fp_set_eq:NN \l_hobby_tempc_fp \l_hobby_tempa_fp
    \fp_sub:Nn \l_hobby_tempc_fp {\l_hobby_tempb_fp}
    \fp_mul:Nn \l_hobby_tempc_fp {\g_hobby_paramc_fp}
    \fp_neg:N \l_hobby_tempc_fp

    \fp_add:Nn \l_hobby_tempc_fp {\l_hobby_tempa_fp}
    \fp_add:Nn \l_hobby_tempc_fp {1}

    \fp_set_eq:NN \l_hobby_tempd_fp \l_hobby_temps_fp

    \fp_add:Nn \l_hobby_tempd_fp {2}

    \fp_div:Nn \l_hobby_tempd_fp {\l_hobby_tempc_fp}

   \prop_put:Nnx \l_hobby_rho_prop {##1} {\fp_to_tl:N \l_hobby_tempd_fp}

  }
\prop_show:N \l_hobby_angles_prop
\prop_show:N \l_hobby_theta_prop
  \prop_map_inline:Nn \l_hobby_theta_prop {
    \prop_get:NnN \l_hobby_angles_prop {##1} \l_tmpa_tl
    \fp_set:Nn \l_hobby_tempc_fp {\l_tmpa_tl}
    \fp_add:Nn \l_hobby_tempc_fp {##2}
    \fp_sin:Nn \l_hobby_tempd_fp {\l_hobby_tempc_fp}
    \fp_cos:Nn \l_hobby_tempc_fp {\l_hobby_tempc_fp}
    \prop_get:NnN \l_hobby_rho_prop {##1} \l_tmpa_tl
    \fp_mul:Nn \l_hobby_tempd_fp {\l_tmpa_tl}
    \fp_mul:Nn \l_hobby_tempc_fp {\l_tmpa_tl}
    \prop_get:NnN \l_hobby_distances_prop {##1} \l_tmpa_tl
    \fp_mul:Nn \l_hobby_tempd_fp {\l_tmpa_tl}
    \fp_mul:Nn \l_hobby_tempc_fp {\l_tmpa_tl}
    \fp_div:Nn \l_hobby_tempd_fp {3}
    \fp_div:Nn \l_hobby_tempc_fp {3}
  \prop_get:NnN \l_hobby_pointsx_prop {##1} \l_tmpa_tl
    \fp_add:Nn \l_hobby_tempc_fp {\l_tmpa_tl}
  \prop_get:NnN \l_hobby_pointsy_prop {##1} \l_tmpa_tl
    \fp_add:Nn \l_hobby_tempd_fp {\l_tmpa_tl}
  \exp_args:NNx \prop_put:Nnx \l_hobby_controla_prop {\int_eval:n {##1 + 1}} {x = \fp_use:N \l_hobby_tempc_fp, y = \fp_use:N \l_hobby_tempd_fp }
  }

  \prop_map_inline:Nn \l_hobby_phi_prop {
    \exp_args:NNx \prop_get:NnN \l_hobby_angles_prop {\int_eval:n {##1 - 1}} \l_tmpa_tl
    \fp_set:Nn \l_hobby_tempc_fp {\l_tmpa_tl}
    \fp_sub:Nn \l_hobby_tempc_fp {##2}
    \fp_sin:Nn \l_hobby_tempd_fp {\l_hobby_tempc_fp}
    \fp_cos:Nn \l_hobby_tempc_fp {\l_hobby_tempc_fp}
    \fp_neg:N \l_hobby_tempc_fp
    \fp_neg:N \l_hobby_tempd_fp
    \prop_get:NnN \l_hobby_sigma_prop {##1} \l_tmpa_tl
    \fp_mul:Nn \l_hobby_tempc_fp {\l_tmpa_tl}
    \fp_mul:Nn \l_hobby_tempd_fp {\l_tmpa_tl}
    \exp_args:NNx \prop_get:NnN \l_hobby_distances_prop {\int_eval:n {##1 - 1}} \l_tmpa_tl
    \fp_mul:Nn \l_hobby_tempc_fp {\l_tmpa_tl}
    \fp_mul:Nn \l_hobby_tempd_fp {\l_tmpa_tl}
    \fp_div:Nn \l_hobby_tempc_fp {3}
    \fp_div:Nn \l_hobby_tempd_fp {3}
    \prop_get:NnN \l_hobby_pointsx_prop {##1} \l_tmpa_tl
    \fp_add:Nn \l_hobby_tempc_fp {\l_tmpa_tl}
    \prop_get:NnN \l_hobby_pointsy_prop {##1} \l_tmpa_tl
    \fp_add:Nn \l_hobby_tempd_fp {\l_tmpa_tl}
    \prop_put:Nnx \l_hobby_controlb_prop {##1} {x = \fp_use:N \l_hobby_tempc_fp, y = \fp_use:N \l_hobby_tempd_fp }
  }
}
%    \end{macrocode}
% \end{macro}
%
% \begin{macro}{\hobbyinit}
% Initialise the settings for Hobby's algorithm
%    \begin{macrocode}
\NewDocumentCommand \hobbyinit {m m} {
\hobby_set_cmds:nn#1#2
\prop_clear:N \l_hobby_points_prop
\prop_clear:N \l_hobby_pointsx_prop
\prop_clear:N \l_hobby_pointsy_prop
\prop_clear:N \l_hobby_angles_prop
\prop_clear:N \l_hobby_distances_prop
\prop_clear:N \l_hobby_tensionout_prop
\prop_clear:N \l_hobby_tensionin_prop
\prop_clear:N \l_hobby_tria_prop
\prop_clear:N \l_hobby_trib_prop
\prop_clear:N \l_hobby_tric_prop
\prop_clear:N \l_hobby_trid_prop
\prop_clear:N \l_hobby_psi_prop
\prop_clear:N \l_hobby_psid_prop
\prop_clear:N \l_hobby_theta_prop
\prop_clear:N \l_hobby_phi_prop
\prop_clear:N \l_hobby_sigma_prop
\prop_clear:N \l_hobby_rho_prop
\prop_clear:N \l_hobby_alpha_prop
\prop_clear:N \l_hobby_controla_prop
\prop_clear:N \l_hobby_controlb_prop

  \int_set:Nn \l_hobby_npoints_int {-1}
  \fp_set_eq:NN \l_hobby_inang_fp \c_undefined_fp
  \fp_set_eq:NN \l_hobby_outang_fp \c_undefined_fp
  \fp_set_eq:NN \l_hobby_incurl_fp \c_one_fp
  \fp_set_eq:NN \l_hobby_outcurl_fp \c_one_fp
}
%    \end{macrocode}
% \end{macro}
%
% \begin{macro}{\hobbyaddpoint}
%    \begin{macrocode}
\NewDocumentCommand \hobbyaddpoint { m } {
    \keys_set:nn { hobby/read in all }
    {
      tension~out,
      tension~in,
      #1
    }
    \int_incr:N \l_hobby_npoints_int
    \exp_args:NNo \prop_put:Nnx \l_hobby_tensionout_prop {\int_use:N \l_hobby_npoints_int } {\fp_to_tl:N \l_hobby_tempc_fp}
    \exp_args:NNo \prop_put:Nnx \l_hobby_tensionin_prop {\int_use:N \l_hobby_npoints_int } {\fp_to_tl:N \l_hobby_tempd_fp}
    \exp_args:NNo \prop_put:Nnx \l_hobby_points_prop {\int_use:N \l_hobby_npoints_int } {x = \fp_use:N \l_hobby_tempa_fp, y = \fp_use:N \l_hobby_tempb_fp }
    \exp_args:NNo \prop_put:Nnx \l_hobby_pointsx_prop {\int_use:N \l_hobby_npoints_int } {\fp_to_tl:N \l_hobby_tempa_fp}
    \exp_args:NNo \prop_put:Nnx \l_hobby_pointsy_prop {\int_use:N \l_hobby_npoints_int } {\fp_to_tl:N \l_hobby_tempb_fp}
}
%    \end{macrocode}
% \end{macro}
%
% \begin{macro}{\hobbysetparams}
%    \begin{macrocode}
\NewDocumentCommand \hobbysetparams { m } {
  \keys_set:nn { hobby / read in params }
  {
    #1
  }
}
%    \end{macrocode}
% \end{macro}
%
% \begin{macro}{\hobbygenpath}
%    \begin{macrocode}
\cs_new:Nn \hobby_set_cmds:nn {
  \cs_set_eq:NN \hobby_moveto:n #1
  \cs_set_eq:NN \hobby_curveto:nnn #2
}
\tl_new:N \l_tmpc_tl
\NewDocumentCommand \hobbygenpath { } {
  \hobby_genpath:
  \prop_get:NnN \l_hobby_points_prop {0} \l_tmpa_tl
  \exp_args:No \hobby_moveto:n {\l_tmpa_tl}
  \prg_stepwise_inline:nnnn {1} {1} {\l_hobby_npoints_int} {
    \prop_get:NnN \l_hobby_controla_prop {##1} \l_tmpa_tl
    \prop_get:NnN \l_hobby_controlb_prop {##1} \l_tmpb_tl
    \prop_get:NnN \l_hobby_points_prop {##1} \l_tmpc_tl
    \exp_args:Nooo \hobby_curveto:nnn {\l_tmpa_tl} {\l_tmpb_tl} {\l_tmpc_tl}
}
}
%    \end{macrocode}
% \end{macro}
%    \begin{macrocode}
\ExplSyntaxOff
%    \end{macrocode}
% \iffalse
%</hobby>
% \fi
%
% \subsection{PGF Library}
%
% \iffalse
%<*pgflibrary>
% \fi
% 
%    \begin{macrocode}
\input{hobby.code.tex}

\pgfkeys{
  /pgf/hobby/.is family,
  /pgf/hobby/.cd,
  x/.code={\pgf@x=#1cm},
  y/.code={\pgf@y=#1cm},
}

\def\hobby@curveto#1#2#3{
  \pgfpathcurveto{\hobby@topgf{#1}}{\hobby@topgf{#2}}{\hobby@topgf{#3}}%
}

\def\hobby@topgf#1{%
    \pgfqkeys{/pgf/hobby}{#1}
}

%    \end{macrocode}
% \iffalse
%</pgflibrary>
% \fi
%
% \subsection{TikZ Library}
%
% \iffalse
%<*tikzlibrary>
% \fi
% 
%    \begin{macrocode}
%
\usepgflibrary{hobby}

\tikzset{
  curve through/.style={
    to path={
      \pgfextra{
        \curvethrough{(\tikztostart) .. #1 .. (\tikztotarget)}
      }
    }
  },
  tension in/.code = {},
  tension out/.code = {},
}
%    \end{macrocode}
% \begin{macro}{\curvethrough}
%    \begin{macrocode}
\newcommand\curvethrough[2][]{%
  \hobbyinit\pgfutil@gobble\hobby@curveto
  \hobbysetparams{#1}%
  \hobby@processpts{#2}%
}

\newcommand\hobby@processpts[1]{%
  \pgfutil@in@{..}{#1}%
  \ifpgfutil@in@%
    \hobby@getonepoint #1 \relax
    \let\hobby@next=\hobby@processpts
  \else
    \def\hobby@pt{#1}%
    \def\hobby@rest{}%
    \let\hobby@next=\hobbygenpath
  \fi
  \let\tikz@scan@point@options=\pgfutil@empty
  \expandafter\tikz@scan@one@point\expandafter\pgfutil@firstofone\hobby@pt\relax
  \pgfmathsetmacro\hobby@x{\the\pgf@x/1cm}%
  \pgfmathsetmacro\hobby@y{\the\pgf@y/1cm}%
  \expandafter\hobbyaddpoint\expandafter{\tikz@scan@point@options,x = \hobby@x, y = \hobby@y}%
  \expandafter\hobby@next\expandafter{\hobby@rest}%
}
%    \end{macrocode}
% \end{macro}
%
% \begin{macro}{\hobby@onepoint}
%    \begin{macrocode}
\def\hobby@getonepoint#1..#2\relax{%
  \def\hobby@pt{#1}%
  \def\hobby@rest{#2}%
}
%    \end{macrocode}
% \end{macro}
% \iffalse
%</tikzlibrary>
% \fi
%
%\Finale